\documentclass{article}
\usepackage[margin=2.5cm]{geometry}
\usepackage[T2A]{fontenc}
\usepackage[utf8]{inputenc}
\usepackage[russian]{babel}
\usepackage{amssymb, amsmath}
\title{Алгоритм петерсона}
\vspace{-2cm}

\begin{document}
\Large

Нам нужно доказать, что паросочетание размера $|L|$ существует $\iff$ для любого A, подмножества L, верно: $|A| \leqslant |N(A)|$. 

$\implies$:\\
Рассмотрим некоторое A --- подмножество L. Существует паросочетание размера $|L|$, значит каждая вершина $a_i$ из A соединена ребром из этого паросочетания с вершиной $b_i$ из R, причём так как $(a_i, b_i)$ --- рёбра паросочетания, то все $b_i$ различны. Так как $b_i \in N(A)$, то $|N(A)| \ \geqslant |A|$.

$\impliedby$:\\
Рассмотрим максимальное паросочетание H, пусть его размер меньше $|L|$.  Ориентируем рёбра в графе: если ребро $(a, b)$ ($a \in A, b \in B$) есть в паросочетании H, то ориентируем его как $(a \leftarrow b)$, иначе $(a \rightarrow b)$. Пусть $a$ --- вершина из L, не вошедшая в паросочетание. Запустим из неё dfs (в графе с ориентированными рёбрами), множество посещённых вершин обозначим за X, множество посещённых вершин из левой доли за $X_L$, правой --- $X_R$. Если в $X_R$ есть вершина не вошедшая в H, то мы нашли удлиняющую цепь, противоречие с выбором H. Значит все вершины из $X_R$ входят в H, значит из каждой вершины $r_i$ из $X_R$ по ребру из R в L (которое входит в H) мы пройдём в вершину $l_i \in X_L$, причём все $l_i$ различны и ни одна из них не совпадает с $a$. Также понятно, что так как в вершины $X_L$ мы приходим только по обратным рёбрам из $X_R$, то $X_L \cup \{a\} == X_R$. Значит $|X_L| = |X_R| + 1$, но $|X_L| \leqslant |N(X_L)| = |X_R|$. Противоречие, значит размер H как раз и равен $|L|$.
\end{document}
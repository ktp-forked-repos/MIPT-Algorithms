\documentclass{article}
\usepackage{mathtext}
\usepackage[T2A]{fontenc}
\usepackage[utf8]{inputenc}
\usepackage[russian]{babel}
\usepackage{amsmath, amssymb, amsthm}

\date{}
\title{Задача 2 (на 30.11.2015)}

\begin{document}
\Large
\maketitle
Воспользуемся декартовым деревом по неявному ключу. В вершине дерева будем хранить ключ, размер поддерева $size$, сумму $k$-ых степеней в поддереве $0 \leqslant k \leqslant 4$, переменную $toadd$, содержащую число, которое надо прибавить ко всем вершинам в поддереве, флаг $isAssigned$ и переменную $toAssign$, (если флаг истинен то это означает, что всем вершинам в поддереве надо присвоить значение $assign$). 

Чтобы реализовать операции $split$ и $merge$ нам надо понять, как проталкиваются значения переменных $toadd$ и $assign$, и как пересчитываются значения остальных переменных. В начале каждой операции с вершиной будем делать следующее:

1. Если флаг $isAssigend$ истинен, то обнуляем его, обнуляем переменную $toadd$ в двух потомках (если они есть, конечно), присваиваем 1 переменной $isAsigned$ в потомках и присваиваем нашему ключу и переменным $assign$ в потомках значение нашей переменной $assign$. Значения $k$-ых степеней рассчитываем зная размер поддерева и значение переменной $assign$.

2. Если $toadd$ не равно нулю, то прибавляем к переменным $toadd$ в потомках значение нашей переменной $toadd$, к сумме первых степеней поддерева прибавляем $size \cdot toadd$. Пусть $sum$ и $sum_2$ - старые суммы первых и вторых степеней поддерева. Тогда $sum_2=x^2+y^2+\ldots$. Если ко всем ключам надо прибавить $toadd$, то новая сумма $$sum_2'=(x+toadd)^2+(y+toadd)^2+\ldots=$$
$$=(x^2+y^2+\ldots)+2 \cdot toadd \cdot (x+y+\ldots)+size\cdot toadd^2=$$
$$=sum_2+2 \cdot toadd \cdot sum + size \cdot toadd^2$$
Таким образом значения суммы $k$-ых степеней в поддереве вычисляем зная размер поддерева и старую сумму в поддереве всех степеней $\leqslant k$.

Мы научились делать $split$ и $merge$. Выразим через них другие операции.

* Сумма $k$-ых степеней на отрезке: два split'а, получаем значение нужной суммы из среднего получившегося дерева, два merge'а.

* Вставка элемента: split, создаём новое дерево из одной вершины, два merge'а.

* Присвоение элемента на отрезке: два split'а, присваиваем нужное значение переменной $assign$ в среднем получившемся дереве, два merge'а.

* Прибавление на отрезке: два split'а, присваиваем нужное значение переменной $toadd$ в среднем получившемся дереве, два merge'а.

* Циклический сдвиг на отрезке: два split'а, получили дерево X --- отрезок, который нам надо циклически сдвинуть. Рассмотрим случай циклического сдвига влево на $k$. Делаем такой merge, чтобы получить два дерева L размера $k$ и R, далее делаем merge деревьев R и L (то есть в обратном порядке), получили новое дерево X --- циклически сдвинутый отрезок, далее два merge'а.

Это были пункты а) и б)

Теперь рассмотрим пункт в)

Разобьём наш массив $array[n]$ на пять массивов $parts[5][k]$: $\forall i\in[0 \ldots n)\ array[i] = parts[i\ mod\ 5][i\ div\ 5]$. По каждому из этих пяти массивов построим декартово дерево как в предыдущем пункте. Рассмотрим некоторые операции:

* Циклический сдвиг на отрезках длины 5: в каждом из пяти массивов делаем два split'а, получили массив из пяти деревьев: $Trees[5]$, $i$-ое дерево содержит $i$-ые элементы каждого отрезка из пяти элементов, соответственно чтобы сделать циклический сдвиг каждого отрезка длины пять надо циклически сдвинуть массив $Trees[5]$. Далее для каждого дерева из массива $Trees[5]$ (уже циклически сдвинутого) делаем два split'а с соответствующими деревьями (которые получились в результате двух merge'ей).

* Вставка элемента x: рассмотрим участок массива $array$, в который надо вставить элемент: 
$$\ldots a_1b_1c_1d_1e_1\ a_2b_2c_2d_2e_2\ a_3b_3c_3d_3e_3 \ldots$$ $a_1,a_2,a_3$ --- элементы массива $parts[0][k]$\\ 
$b_1,b_2,b_3$ --- элементы массива $parts[1][k]$\\
$\ldots$\\
Пусть нам надо вставить элемент между $b_2$ и $c_2$. В каждом из пяти массивов $parts[5]$ делаем split, так чтобы все элементы, которые после вставки x в массив $array[n]$ окажутся справа от x оказались в правом дереве split'а. Вот что получится в нашем случае ($LTrees[i]$ и $RTrees[i]$ --- результат split'а дерева построенного на массиве $parts[i]$)
$$LTrees[0] = \ldots a_1,a_2\quad RTrees[0] = a_3\ldots$$
$$LTrees[1] = \ldots b_1,b_2\quad RTrees[1] = b_3\ldots$$
$$LTrees[2] = \ldots c_1\quad RTrees[2] = c_2,c_3\ldots$$
$$LTrees[3] = \ldots d_1\quad RTrees[3] = d_2,d_3\ldots$$
$$LTrees[4] = \ldots e_1\quad RTrees[4] = e_2,e_3\ldots$$
После вставки массив $array$ будет выглядеть следующим образом:
$$\ldots a_1b_1c_1d_1e_1\ a_2b_2xc_2d_2\ e_2a_3b_3c_3d_3\ e_3 \ldots$$
Поэтому сделаем split дерева из одного элемента x с соответствующим деревом из $RTrees[5]$, в нашем случае это будет $RTrees[1]$:
$$RTrees[1]' = x,b_3\ldots$$
Пусть после предыдущего шага элемент $x$ находится в дереве $RTrees[i]$. Тогда теперь нам надо сделать merge следующих деревьев: 
$$LTrees[i+1]\ c\ RTrees[i]$$
$$LTrees[i+2]\ c\ RTrees[i+1]$$
$$\ldots$$
$$LTrees[0]\ c\ RTrees[4]$$
Вот что получится в нашем случае ($Trees[i]$ --- результат merge'а дерева $LTrees[i]$ и $RTrees[(5+i-1)\ mod\ 5]$):
$$Trees[0] = \ldots a_1,a_2,e_2,e_3\ldots$$
$$Trees[1] = \ldots b_1,b_2,a_3\ldots$$
$$Trees[2] = \ldots c_1,x,b_3\ldots$$
$$Trees[3] = \ldots d_1,c_2,c_3\ldots$$
$$Trees[4] = \ldots e_1,d_2,d_3\ldots$$
Заметим, что получившиеся пять деревьев $Trees[5]$ --- именно такие, как если бы мы их строили по новому массиву $array$ (с элементом $x$).

\end{document}
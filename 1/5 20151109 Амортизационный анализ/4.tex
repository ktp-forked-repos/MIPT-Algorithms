\documentclass{article}
\usepackage{mathtext}
\usepackage[T2A]{fontenc}
\usepackage[utf8]{inputenc}
\usepackage[russian]{babel}
\usepackage{amsmath, amssymb}
\usepackage{mathtools}

\title{Задача 4}
\date{}

\DeclarePairedDelimiter\ceil{\lceil}{\rceil}

\begin{document}
\maketitle
\LARGE

{\centering{\textbf{Реализация}}\\}
Будем хранить наши элементы в векторе в порядке добавления.\\
\textbf{Insert(x)}

Добавим $x$ в конец вектора. (операция $push\_back$)\\
\textbf{DeleteLargerHalf}

Найдём медиану нашего вектора методом медианы медиан. Пройдёмся по вектору и удалим (алгоритм удаления чуть позже) все элементы большие медианы, пусть их было $x$ штук. По определению медианы $x<\ceil{\frac{n}{2}}$. Ещё раз пройдемся по вектору и удалим $\ceil{\frac{n}{2}}-x$ элементов равных медиане. Таким образом мы удалили $\ceil{\frac{n}{2}}$ элементов, что и требовалось. 

Как мы будем удалять: меняем элемент который хотим удалить с последним элементом в векторе и делаем $pop\_back$.

\newpage
Проведем {\centering{\textbf{амортизационный анализ}}} методом усреднения. Изначально в структуре $0$ элементов, после выполнения $n$ операций в ней останется $k$ элементов. Пусть среди $n$ операций было $m$ операций $DeleteLargerHalf$. Обозначим за $a_i$ число элемнтов, удаленных на $i$-ой операции $DeleteLargerHalf$. Тогда всего было $(n-m)$ операций добавления элементов. В конце осталось $k$ элементов, запишем это:
$$(n-m)-\sum_{i=1}^{m}a_i=k$$
$$\sum_{i=1}^{m}a_i=n-m-k$$

Время выполнения операции $Insert(x)$ --- амортизированно $O(1)$. 

Время выполнения операции $DeleteLargerHalf$ складывается из времени поиска медианы, времени двух проходов по вектору и времени выполнения $a_i$ операций $pop\_back$. Если мы удаляем $a_i$ элементов, то до удаления в структуре было $a_i\cdot2\pm1$ элементов, то есть $\Theta(a_i)$. Медиану мы ищем за $O(число\ элементов\ в\ структуре)=O(a_i)$, по массиву мы проходимся за те же \\$\Theta(число\ элементов\ в\ структуре)=\Theta(a_i)$, операции $pop\_back$ мы выполняемя амортизированно за $O(1)$. Значит все операции  $pop\_back$ мы выполним амортизированно за $a_i\cdot O(1)=\Theta(a_i)$ и амортизированное время выполнения $DeleteLargerHalf$ равно $\Theta(a_i)+O(a_i)+\Theta(a_i)=\Theta(a_i)$

Таким образом суммарное время выполнения всех $n$ операций равно 
$$(n-m)\cdot O(1)+\sum_{i=1}^{m}\Theta(a_i)=$$
$$=\Theta(n-m)+\Theta(\sum_{i=1}^{m}a_i)=$$
$$=\Theta(n-m)+\Theta(n-m-k)=$$
$$=\Theta(2n-2m-k)=\Theta(n)$$
Значит амортизационная стоимость выполнения одной операции равна $\frac{\Theta(n)}{n}=O(1)$

\end{document}
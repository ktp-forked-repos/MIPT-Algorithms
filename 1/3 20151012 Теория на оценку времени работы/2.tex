\documentclass[12pt]{article}
\usepackage{mathtext}
\usepackage[T2A]{fontenc}
\usepackage[utf8]{inputenc}
\usepackage[russian]{babel}
\usepackage{amsmath, amssymb}
\usepackage{enumerate}
\title{Задача 2}
\date{}

\begin{document}
	\large
	\maketitle
	\begin{enumerate}[a.]
		\item
			$T(n) = 5T(n/4) + n$\\
			Обозначим $f(n) = n,\; g(n) = n^{\log_4 5}$\\
			Сравним $f(n)\; и\; g(n)$\\
			$f(n) = O(n^{(\log _4 5)-\varepsilon}),\; \varepsilon>0 $\\
			Значит $T(n) = \Theta(g(n)) = \Theta(n^{\log_4 5})$
		\item
			$T(n) = T(4n/5) + 1$\\
			Обозначим $f(n) = 1,\; g(n) = n^{\log_{5/4} 1} = n^0 = 1$\\
			Сравним $f(n)\; и\; g(n)$\\
			$f(n) = \Theta(g(n))$\\
			Значит $T(n) = \Theta(g(n)\log n) = \Theta(\log n)$
		\item
			$T(n) = 7T(n/8) + n\log n$\\
			Обозначим $f(n) = n\log n,\; g(n) = n^{\log_8 7}$\\
			Сравним $f(n)\; и\; g(n)$\\
			$f(n) = \Omega(n^{(\log_8 7) + \varepsilon}),\; \varepsilon>0$\\
			Проверим, что существует такое $c < 1$, что
			$$7f(\frac{n}{8}) \le cf(n)$$
			Возьмём $c = \frac{7}{8}$. Тогда
			$$cf(n) - 7f(\frac{n}{8}) = \frac{7n}{8}\log n - \frac{7n}{8}\log\frac{n}{8} = \frac{7n}{8}\log 8 \ge 0$$
			Значит $T(n) = \Theta(f(n)) = \Theta(n\log n)$
	\end{enumerate}
\end{document}
\documentclass[12pt]{article}
\usepackage{mathtext}
\usepackage[T2A]{fontenc}
\usepackage[utf8]{inputenc}
\usepackage[russian]{babel}
\usepackage{amsmath, amssymb, amsthm}
\usepackage{enumerate}
\usepackage{mathtools}

%\newcommand{\Tceil}[1]{T(\ceil{#1})}
\newcommand{\logn}{\log n}
\newcommand{\sqrtn}{\sqrt n}
\newcommand{\logsn}{\log\sqrtn}
\newcommand{\e}{\log_2 3}

\begin{document}
	\large
	%\LARGE
	%\huge
	Найдём время работы алгоритма поиска медианы \\
	при делении элементов на группы по $2k+1$ элементов. \\
	Оно складывается из:
	\begin{enumerate}
		\item
			Времени нахождения медианы в каждой \\
			из частей по $(2k+1)$ элементов:
			$$T_1(n)=\frac{n}{2k+1}(2k+1)^2=\Theta(n)$$
		\item
			Времени нахождения медианы медиан:
			$$T_2(n)=T(\frac{n}{2k+1})$$
		\item
			Времени разделение массива на две части:
			$$T_3(n) = \Theta(n)$$
		\item
			Времени рекурсивного вызова от одной из частей.\\
			Так как медиана медиан не меньше чем как минимум
			$$\frac{n(k+1)}{2(2k+1)}$$ элементов, 
			то каждая часть содержит не более чем
			$$n-\frac{n(k+1)}{2(2k+1)}=\frac{3k+1}{2(2k+1)}n$$ элементов, то есть
			$$T_4(n)=T(\frac{3k+1}{2(2k+1)}n)$$
	\end{enumerate}
	\newpage
	Таким образом,
	$$T(n)=T_1(n)+T_2(n)+T_3(n)+T_4(n)$$
	$$T(n)=T(\frac{n}{2k+1})+T(\frac{3k+1}{2(2k+1)}n)+dn$$
	Посмотрим, для каких $k$ верно $T(n) = O(n)$\\
	Для этого должно выполняться
	$$\begin{cases}
			T(\frac{n}{2k+1}) \leq c(\frac{n}{2k+1})\\
			T(\frac{3k+1}{2(2k+1)}n) \leq c(\frac{3k+1}{2(2k+1)}n)\\
		\end{cases}
	\implies T(n) \leq cn$$
	то есть
	$$c\frac{n}{2k+1}+c\frac{3k+1}{2(2k+1)}n+dn \leq cn$$
	$$c\frac{(4k+2)-2-(3k+1)}{2(2k+1)} \geq d$$
	$$c\frac{k-1}{2(2k+1)} \geq d$$
	Заметим, что при $k > 1$ мы всегда сможем подобрать такое $c$,\\
	Значит при $k>1$ алгоритм будет работать за линейное время,\\
	а при $k=1$ не за линейное.
\end{document}
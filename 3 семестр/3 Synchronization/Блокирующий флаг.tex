\documentclass{article}
\usepackage[T2A]{fontenc}
\usepackage[utf8]{inputenc}
\usepackage[russian]{babel}

\usepackage{listings}
\usepackage{solarized-light}
\lstset{language=C++}
\definecolor{light-gray}{gray}{0.95}
\newcommand{\code}[1]{\colorbox{light-gray}{\texttt{#1}}}

\begin{document}

\begin{lstlisting}
class blocking_flag {
public:
	blocking_flag() : ready_(false) {}

	void wait() {
		std::unique_lock<std::mutex> lock(mtx_);
		while (!ready_.load()) {
			ready_cond_.wait(lock);
		}
	}

	void set() {
		ready_.store(true);
		ready_cond_.notify_all();
	}

private:
	std::atomic<bool> ready_;
	std::mutex mtx_;
	std::condition_variable ready_cond_;
}
\end{lstlisting}

Нам нужно оценить корректность этого кода. 
Покажем что при некотором порядке исполнения инструкций код выполнится некорректно.
Рассмотрим код из примера:

\begin{lstlisting}
#include <thread>   
#include <iostream>   

int main() {
	blocking_flag f;

	std::thread t(
			[&f]() {
				f.wait();
				std::cout << "ready!" << std::endl;
			}
	);

	f.set();
	t.join();

	return 0;
}
\end{lstlisting}

И следующую последовательность инструкций:\\
1. Главный поток прерывается перед строчкой f.set()\\
2. Поток t вызывает f.wait(), захватывает mutex, проверяет условие while, заходит в while и прерывается\\
3. Главный поток вызывает f.set(), выполняет ready\_.store(true) и ready\_cond\_.notify\_all()\\
4. Поток t выполняет ready\_cond\_.wait(lock) и засыпает, так как notify\_all() был вызван до wait()\\
5. Главный поток вызывает f.join()\\
6. Оба потока неограниченно долго находятся в системных вызовах ожидания, которые при нормальных условиях (то есть без spurious wakeup) не могут завершиться.

\end{document}
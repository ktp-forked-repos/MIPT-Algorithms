\documentclass{article}
\usepackage{mathtext}
\usepackage[T2A]{fontenc}
\usepackage[utf8]{inputenc}
\usepackage[russian]{babel}
\usepackage{amsmath, amssymb}

\newcommand{\ci}[1][i]{capacity_{#1}}
\newcommand{\cpi}{\ci[i-1]}
\newcommand{\si}[1][i]{size_{#1}}
\newcommand{\spi}{\si[i-1]}
\newcommand{\cj}{\ci[j]}
\newcommand{\sj}{\si[j]}

\title{Задача 2}
\date{}

\begin{document}

\maketitle

\Large
{\centering{\textbf{Реализация}}\\}
В классе будут храниться текущее число элементов в деке ($size$), массив размера $capacity$ и индекс первого элемента дека $iFirst$. Изначально $capacity$ равно единице. В каждый момент времени (если дек не пустой) будет выполняться
$$size \leqslant capacity \leqslant size*4$$
Элементы дека будут находиться (по порядку, начиная с первого) в массиве на следующих индексах: $[iFirst, iFisrt+size)$, если $iFisrt + size \leqslant capacity$ и на \\$[iFirst,capacity)\ и\ [0,size-(capacity-iFirst))$ иначе.

\begin{itemize}
\item \textbf{push\_back, push\_front}\\
	Если $size\neq capacity$, то записываем в соответствующую ($(iFirst-1+capacity)\ mod\ capacity$ в случае pop и $(iFirst+size)\ mod\ capacity$ в случае push) ячейку массива элемент, увеличиваем значение $size$ на единицу и обновляяем $iFirst$ если нужно (то есть в случае операции $push\_front$).
	
	Иначе создаём новый массив размера ${capacity*2}$, копируем соответствующие $size$ элементов из старого массива в новый (на первые $size$ позиций), делаем присваивание $iFirst=0$ старый массив удаляем, далее делаем аналогично случаю ${size \neq capacity}$.
\item \textbf{pop\_back, pop\_front}\\
	Вначале уменьшим значение $size$ на единицу и обновим значение $iFirst$ если нужно (то есть в случае операции $pop\_front$). Далее, если $capacity > size*4$, то создаём новый массив размера ${\frac{capacity}{2}}$, копируем соответствующие $size$ элементов из старого массива в новый (на первые $size$ позиций), старый массив удаляем и делаем присваивание $iFirst=0$.
\item \textbf{Доступ по идексу}\\
	Возвращаем соответствующий элемент массива.
\end{itemize}

{\centering{\textbf{Амортизационный анализ}}\\}
	Полностью (символ в символ) повторяет таковой для вектора, поэтому я не стал копировать сюда то что было в векторе.
\end{document}
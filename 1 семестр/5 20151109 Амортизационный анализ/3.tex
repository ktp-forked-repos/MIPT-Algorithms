\documentclass{article}
\usepackage{mathtext}
\usepackage[T2A]{fontenc}
\usepackage[utf8]{inputenc}
\usepackage[russian]{babel}
\usepackage{amsmath, amssymb}

\title{Задача 3}
\date{}

\begin{document}
\maketitle
\LARGE

Реальное время работы такого алгоритма равно $O(k)$. 
Например рассмотрим следующий шаг алгоритма: все биты счётчика равны единице, тогда в цикле выполнится $\Theta(k)$ действий.

Докажем, что учётное время работы равно $O(1)$. Обозначим за $z_i$ число битов счётчика, равных единице после $i$-ой операции. 

Введём потенциалы: $f_i=z_i\cdot c,\ c>0$. Изначально все биты счётчика нулевые, поэтому $f_0=0$. Также по нашему определению $f_i\geqslant0$.

Пусть $t_i$ --- реальное время выполнения $i$-ой операции, а $a_i$ --- учётное время. 
$$a_i = t_i+f_i-f_{i-1}$$

Докажем, что $a_i=O(1)$:\\
Цикл $while (i < A.length() \&\& A[i] == 1)$ будет итерироваться пока $A[i]$ не станет равным нулю. Пусть $k$ --- индекс младшего нулевого бита до $i$-ой операции. Тогда $t_i=\Theta(k)$, $z_{i-1}-k=z_i-1 \implies f_i-f_{i-1}=-k\cdot c+c$. Значит $a_i=\Theta(k)-k\cdot c + c$. Мы всегда можем подобрать $c$ чтобы перекрыть константу в $\Theta$, поэтому $a_i=с=O(1)$, то есть амортизационное время работы алгоритма равно $O(1)$.

Пусть теперь есть ещё операция \textbf{Decrement}. Рассмотрим следующую последовательность из $2n$ операций (изначально все биты счётчика нулевые): каждая операция с нечётным индексом (нумерация с 1) - это Decrement, с чётным - это Increment. Тогда реальное время выполнения каждой операции будет $\Theta(k)$, значит и учётное время работы будет тоже $\Theta(k)$.

\end{document}
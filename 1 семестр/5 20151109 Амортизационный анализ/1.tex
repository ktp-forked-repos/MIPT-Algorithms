\documentclass{article}
\usepackage[utf8]{inputenc}
\usepackage[russian]{babel}
\usepackage{amsmath, amssymb}

\newcommand{\ci}[1][i]{capacity_{#1}}
\newcommand{\cpi}{\ci[i-1]}
\newcommand{\si}[1][i]{size_{#1}}
\newcommand{\spi}{\si[i-1]}
\newcommand{\cj}{\ci[j]}
\newcommand{\sj}{\si[j]}

\title{Задача 1}
\date{}

\begin{document}
\maketitle
\Large

{\centering{\textbf{Реализация}}\\}
В классе будут храниться текущее число элементов в векторе ($size$) и массив размера $capacity$. Изначально $capacity$ равно единице. В каждый момент времени (если вектор не пустой) будет выполняться
$$size \leqslant capacity \leqslant size\cdot4$$

\begin{itemize}
\item \textbf{push\_back}\\
	Если $size\neq capacity$, то записываем в соответствующую ячейку массива элемент и увеличиваем значение $size$ на единицу.
	
	Иначе создаём новый массив размера ${capacity\cdot2}$, копируем первые $size$ элементов из старого массива в новый, старый массив удаляем, далее делаем аналогично случаю ${size \neq capacity}$.
\item \textbf{pop\_back}\\
	Вначале уменьшим значение $size$ на единицу. Далее, если $capacity > size\cdot4$, то создаём новый массив размера ${\frac{capacity}{2}}$, копируем первые $size$ элементов из старого массива в новый, старый массив удаляем.
\item \textbf{Доступ по идексу}\\
	Возвращаем соответствующий элемент массива.
\end{itemize}
		
\newpage
{\centering{\textbf{Амортизационный анализ}}\\}
\textbf{Метод потенциалов}\\
	Обозначим $\si$ и $\ci$ - значения $size$ и $capacity$ в нашем классе после выполнения $i$-ой операции. Введём потенциалы $f$: 
	$$\begin{cases}
		f_0=0\\
		f_i=|\si\cdot2-\ci|, i>0
	\end{cases}$$
	Пусть $t_i$ - реальное время выполнения $i$-ой операции, а $a_i$ - учётное время. 
	$$a_i = t_i+f_i-f_{i-1}$$
	
	Докажем, что $a_i=O(1)$:
	
	Если в результате $i$-ой операции размер массива не изменился, то есть $t_i=O(1)$ и $\ci=\cpi$, то
	$f_i-f_{i-1}=|\si\cdot2-\ci|-|\spi\cdot2-\cpi|=O(1)$.
	Значит в этом случае $$a_i=t_i+(f_i-f_{i-1})=O(1)+O(1)=O(1)$$
	
	Пусть теперь массив увеличился. Это могло произойти только в операции push\_back, причём должно выполняться\\
	$\begin{cases}
		\cpi=\spi,\\
		\ci=\cpi\cdot2,\\
		\si=\spi+1
	\end{cases}$
	
	Найдём $a_i$:\\
	$f_{i-1}=|\spi\cdot2-\cpi|=|\cpi|=\cpi$\\
	$f_i=|\si\cdot2-\ci|=\\
	=|(\spi+1)\cdot2-\cpi\cdot2|=\\
	=|(\spi-\cpi)\cdot2+O(1)|=|O(1)|=O(1)$\\
	$t_i=\Theta(\cpi)$\\
	$a_i=t_i+f_i-f_{i-1}=\Theta(\cpi)+O(1)-\cpi$\\
	Мы можем домножить каждый потенциал на константу, так чтобы $a_i$ стало равным $O(1)$
	
	Осталось разобрать случай уменьшения массива. Это могло произойти только в операции pop\_back, причём должно выполняться\\
	$\begin{cases}
	\cpi>(\spi-1)\cdot4,\\
	\ci=\frac{\cpi}{2},\\
	\si=\spi-1
	\end{cases}$
	
	Найдём $a_i$:\\
	$\begin{cases}
		\cpi >         (\spi-1)\cdot4 \\
		\cpi \leqslant \spi\cdot4
	\end{cases} \implies\\ 
	\implies \cpi=\spi\cdot4+O(1)\\
	\implies \spi\cdot2=\frac{\cpi}{2}+O(1)$,\\
	$f_{i-1}=|\spi\cdot2-\cpi|=\\
	=|\frac{\cpi}{2}+O(1)-\cpi|=\frac{\cpi}{2}+O(1)$\\
	$f_i=|\si\cdot2-\ci|=\\
	=|\spi\cdot2-\frac{\cpi}{2}+O(1)|=\\
	=|\frac{\cpi}{2}+O(1)-\frac{\cpi}{2}+O(1)|=O(1)$\\
	$a_i=t_i+f_i-f_{i-1}=\\
	=\Theta(\cpi)+O(1)-\frac{\cpi}{2}-O(1)=\\
	=\Theta(\cpi)-\frac{\cpi}{2}+O(1)$\\
	Опять же, мы можем домножить каждый потенциал на константу, так чтобы $a_i$ стало равным $O(1)$
	
	Таким образом учётная стоимость каждой операции составляет $O(1)$
	
\newpage
\textbf{Метод бухгалтерского учёта}\\
	Пусть операции добавления и удаления элемента стоят по 13 монеток, а операция доступа по индексу - одну монетку. Пусть в результате $i$-ой операции изменился размер массива. Тогда $\si=\frac{\ci}{2}+O(1)$. Действительно, если $i$-ая операция это операция добавления элемента, то $$\si=\spi+1=\cpi+1=\frac{\ci}{2}+O(1),$$
	если же $i$-ая операция это операция удаления элемента, то $$\si=\spi-1=\frac{\cpi}{2}+O(1)-1=\frac{\ci}{2}+O(1)$$
	Пусть в результате $i$-ой операции изменился размер массива. Найдём предыдущую операцию, в результате которой изменился размер массива, обозначим её номер за $j$. Тогда\\
	$\begin{cases}
	\cpi=\cj\\
	\si[j]=\frac{\cj}{2}+O(1)
	\end{cases} \implies\\
	\implies \cpi=\sj\cdot2+O(1)$\\
	В результате $i$-ой операции изменился размер массива, значит или $\cpi=\spi$, или $\cpi=\spi\cdot4+O(1)$. В первом случае
	$$|\spi-\sj|=\frac{\cpi}{2}+O(1),$$
	во втором
	$$|\spi-\sj|=\frac{\cpi}{4}+O(1),$$
	то есть между двумя операциями, реальная стоимость которых $O(\cpi)$ было не менее $\frac{\cpi}{4}+O(1)$ операций добавления элемента или операций удаления элемента, реальная стоимость каждой из которых O(1). Каждая из них стоила 13 монеток, одну из монеток мы потратили непосредственно на ту операцию, 12 отложили, поэтому теперь у нас есть $\frac{\cpi}{4}\cdot12=\cpi\cdot3$ монеток, и мы их можем потратить на нашу операцию увеличения/уменьшения массива, то есть мы доказали, что учётная стоимость каждой операции O(1).
	
\end{document}
\documentclass[12pt]{article}
\usepackage{mathtext}
\usepackage[T2A]{fontenc}
\usepackage[utf8]{inputenc}
\usepackage[russian]{babel}
\usepackage{amsmath, amssymb, amsthm}
\usepackage{enumerate}
\usepackage{mathtools}

%\newcommand{\Tceil}[1]{T(\ceil{#1})}
\newcommand{\logn}{\log n}
\newcommand{\sqrtn}{\sqrt n}
\newcommand{\logsn}{\log\sqrtn}
\newcommand{\e}{\log_2 3}

\begin{document}
	%\Large
	\LARGE
	%\huge
	\noindent
	$T(n) = 3T(\sqrtn) + \logn$\\
	Ответ: $T(n) = \Theta((\logn)^{\e})$. Докажем это:
	\begin{itemize}
		\item
			$T(n) = O(\logn^{\e})$, то есть\\
			$T(\sqrtn) \le c(\log\sqrtn)^{\e} + d\logsn$\\
			$\implies T(n) \le c(\logn)^{\e} + d\logn$\\
			Доказательство:\\
			$T(n) = 3T(\sqrtn) + \logn$\\
			$T(n) \le 3(c(\log\sqrtn)^{\e} + d\logsn) + \logn$\\
			$T(n) \le 3c(\frac{1}{2}\logn)^{\e} + \frac{3d\logn}{2} + \logn$\\
			$T(n) \le 3c\frac{(\logn)^{\e}}{2^{\e}} + (\frac{3}{2}d + 1)\logn$\\
			$T(n) \le 3c\frac{(\logn)^{\e}}{3} + (\frac{3}{2}d + 1)\logn$\\
			$T(n) \le c(\logn)^{\e} + (\frac{3}{2}d + 1)\logn$\\
			Возьмём $d = -2$. Тогда $(\frac{3}{2}d + 1) = -2 \implies$\\
			$T(n) \le c(\logn)^{\e} - 2\logn \implies$\\
			$T(n) \le c(\logn)^{\e} + d\logn \qed$
		\item
			Заменив в предыдущем доказательстве все знаки $\le$ на $\ge$ получим\\
			$T(\sqrtn) \ge c(\log\sqrtn)^{\e} + d\logsn \implies$\\
			$T(n) \ge c(\logn)^{\e} + d\logn$, то есть\\
			$T(n) = \Omega(\logn^{\e}) \qed$
	\end{itemize}
\end{document}
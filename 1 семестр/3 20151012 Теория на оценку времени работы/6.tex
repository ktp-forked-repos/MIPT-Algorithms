\documentclass[12pt]{article}
\usepackage{mathtext}
\usepackage[T2A]{fontenc}
\usepackage[utf8]{inputenc}
\usepackage[russian]{babel}
\usepackage{amsmath, amssymb}
\usepackage{enumerate}

\begin{document}
	\large
		Обозначим элемент, полученный в результате работы алгоритма, 
	то есть $i$-ый елемент в отсортированном массиве всех елементов как $m$.
		\par
		Разобьём наш массив на две части: 
	если мы знаем, как элемент соотносится с $m$, 
	то положим его в первую часть, иначе во вторую.
		\par
		Более формально, будем говорить, что элемент $a_u \le a_v$, если мы сравнивали элементы $a_u\ и\ a_v$ и в результате сравнения 	получилось что либо они равны, либо первый строго меньше второго, либо существует такая последовательность индексов $i_1 \dots i_k$, что $i_1 = u,\ i_k = v$ и для всех $2 \le j \le k$ выполняется $a_{i_{j-1}} \le a_{i_j}$. Тогда элемент $x$ лежит в первой части $\iff$ если выполняется либо $x \le m$, либо $m \le x$.
		\par
		Обозначим $L(q)$ - множество элементов, что $x \le q$, \\
		$R(q)$ - множество элементов, что $q \le x$
	
\end{document}
\documentclass{article}
\usepackage[utf8]{inputenc}
\usepackage[T2A]{fontenc}
\usepackage[russian]{babel}
\usepackage{amsmath, amssymb}

\title{Задача 5 (семинар 18.04.2016)}
\author{}
\date{}

\begin{document}
\maketitle

Обозначим $n = V+E+Q$. Сделаем sqrt-декомпозицию по запросам. Итак, у нас есть отрезок из $\sqrt{Q}$ запросов и мы хотим за $O(n)$ их обработать. Для начала получим граф на момент перед первым запросом в нашем отрезке запросов, например следующим образом: до всех запросов заведём мультисет рёбер, изначально в нём находятся рёбра из E. Далее, мы будем по очереди обрабатывать отрезки по $\sqrt{Q}$ запросов, после обработки каждого отрезка будем обновлять наш мультисет. Теперь, у нас есть отрезок запросов и мультисет рёбер. Проитерируемся по мультисету (мы знаем что это $O(n)$), и получим граф в виде списка смежности $g$. 

Теперь удалим из $g$ те рёбра, которые встречаются в нашем отрезке запросов: заметим, что $g_i$ --- отсортированный список, Создадим $h$ --- такой же список смежности, состоящий из рёбер встречающихся в нашем отрезке запросов. Потом сделаем set\_difference. Выделим в полученном графе компоненты связности (dfs), получим граф $g'$, изначально в нём нет рёбер, добавим в него рёбра которые встречаются в запросах и которые были в исходном графе $g$ (на момент до первого запроса). Заметим что в $g'$ $O(\sqrt{n})$ рёбер. Далее по очереди обрабатываем запросы, запрос добавления/удаления ребра делаем просто за $O(\sqrt{n})$, запрос проверки на принадлежность одной компоненте связности --- также за $O(\sqrt{n})$, обычный dfs, ибо в нашем графе $g'$ всегда $O(\sqrt{n})$ рёбер.

\end{document}
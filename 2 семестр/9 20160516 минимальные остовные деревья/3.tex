\documentclass{article}
\usepackage[utf8]{inputenc}
\usepackage[T2A]{fontenc}
\usepackage[russian]{babel}
\usepackage{amsmath, amssymb}

\title{Задача 3 (семинар 16.05.2016)}
\author{}
\date{}

\newcommand{\mimplies}[2]{$\mathbf{#1 \implies #2}$}

\begin{document}
\maketitle

1. Докажем, что если T это минимальный остов, то каждое ребро не лежащее в T имеет максимальный вес среди рёбер цикла, стягиваемого этим ребром.
Пусть $(u, v)$ --- ребро веса $w$, не принадлежащее T, и пусть в цикле, стягиваемым этим ребром, есть ребро $(u', v')$ веса $w'>w$. 
Тогда удалим из T ребро $(u', v')$, добавим ребро $(u, v)$, T останется остовным деревом, но его вес уменьшится, противоречие с минимальностью T.

2. Докажем, что если каждое ребро не лежащее в T имеет максимальный вес среди рёбер цикла, стягиваемого этим ребром (*), то T это минимальный остов.
Выберем среди всех минимальных остовных деревьев такое дерево S, что число его общих с T рёбер максимально.
Рассмотрим случай когда $T \neq S$.
Пусть $(u, v)$ --- ребро минимального веса $w$, не принадлежащее T, но принадлежащее S (или наоборот).

a) $(u, v) \in T$\\
В этом случае $(u, v)$ стягивает некоторой цикл в $S$, и, конечно, не все рёбра этого цикла принадлежат T, иначе бы в T был цикл. 
Пусть $(u', v')$ --- ребро веса $w'$ в цикле, не принадлежащее $T$. 
Конечно, $w' \geq w$ иначе бы было противоречие с выбором ребра $(u, v)$.
Также по пункту (1) $w' \leq w$. Итак, $w' = w$, тогда удалим из S ребро $(u', v')$, добавим ребро $(u, v)$, 
S останется минимальным остовным деревом, но число общих с T вершин возрастёт, противоречие с выбором S.

б) $(u, v) \in S$\\
Аналогично пункту (а) получаем, что в цикле, стягиваемом в T ребром $(u, v)$ есть ребро $(u', v')$ веса $w'=w$ (только вместо пункта 1 используем условие (*)).
Применяя пункт (а) к ребру $(u', v')$ получаем противоречие.

Итак, в обоих случаях мы пришли к противоречию, значит $S=T$, что и требовалось.

\end{document}
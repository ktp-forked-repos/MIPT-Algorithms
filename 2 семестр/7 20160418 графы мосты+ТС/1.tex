\documentclass{article}
\usepackage[utf8]{inputenc}
\usepackage[T2A]{fontenc}
\usepackage[russian]{babel}
\usepackage{amsmath, amssymb}

\title{Задача 1 (семинар 18.04.2016)}
\author{}
\date{}

\newcommand{\mimplies}[2]{$\mathbf{#1 \implies #2}$}
\newcommand{\way}[2]{#1_0 \to #1_1 \to ... \to #1_{n-1} \to #1_#2}
\newcommand{\waya}{\way{a}{n}}
\newcommand{\wayb}{\way{b}{k}}
\newcommand{\mto}[1]{\overset{#1}{\twoheadrightarrow}}
\newcommand{\mtoa}{\mto{1}}
\newcommand{\mtob}{\mto{2}}
\newcommand{\mtoc}{\mto{3}}
\newcommand{\mtod}{\mto{4}}

\begin{document}
\maketitle

Запись ($u \to v)$ означает что существует ребро между $u$ и $v$, а $u \mtoa v$ означает что существует путь из $u$ в $v$, причём мы будем называть его <<путь~1>>.

\mimplies{A}{B}\\
Рассмотрим любые две различные вершины $u$ и $v$. Пусть $(u, u')$ и $(v, v')$ --- любые ребра (такие существуют, ибо граф связный). G не содержит точек сочленения, значит он является вершинно двусвязным, значит существуют два вершинно непересекающихся пути соединяющих концы рёбер $(u, u')$ и $(v, v')$:
$$u= \waya =v$$
$$v'= \wayb =u'$$
Тогда вершины $u$ и $v$ принадлежат простому циклу:
$$u= \waya =v \to v'= \wayb =u' \to u$$

\mimplies{B}{C}\\
Рассмотрим некоторую вершину $w$ и ребро $(u, v)$. Применим B к ($w$ и $u$) и ($w$ и $v$):
$$w \mtoa u \mtob w$$
$$w \mtoc v \mtod w$$
Пусть $x$ --- ближайшая к $u$ вершина лежащая на пути 1, такая что она также лежит на пути 3 или 4. Заметим, что $x$ не может одновременно лежать и на пути 3 и на пути 4, поэтому рассмотрим случай когда $x$ лежит на пути 3, другой случай аналогичен. Тогда вершина $w$ и ребро $(u, v)$ лежат на следующем простом цикле:
$$w \mtoc x \mtoa u \to v \mtod w$$

\mimplies{C}{D}\\
Рассмотрим некоторые рёбра $(a, b)$ и $(c, d)$. Применим C к ($a$~и~$(c, d)$) и ($b$~и~$(c, d)$):
$$c \mtoa a \mtob d \to c$$
$$c \mtoc b \mtod d \to c$$
Пусть $x$ --- ближайшая к $a$ вершина лежащая на пути 1, такая что она также лежит на пути 3 или 4. Заметим, что $x$ не может одновременно лежать и на пути 3 и на пути 4, поэтому рассмотрим случай когда $x$ лежит на пути 3, другой случай аналогичен. Тогда рёбра $(a, b)$ и $(c, d)$ лежат на следующем простом цикле:
$$c \mtoc x \mtoa a \to b \mtod d \to c$$

\mimplies{D}{E}\\
Рассмотрим некоторые вершины $a$ и $b$ и ребро $(c, d)$. Пусть $(a, a')$ и $(b, b')$ --- любые ребра. Применим D к рёбрам ($(a, a')$, $(c, d)$) и ($(b, b')$, $(c, d)$):
$$a \mtoa c \to d \mtob a' \to a$$
$$b \mtoc c \to d \mtod b' \to b$$
Пусть $x$ --- ближайшая к $a$ вершина лежащая на пути 1, такая что она также лежит на пути 3 или 4. Заметим, что $x$ не может одновременно лежать и на пути 3 и на пути 4, поэтому рассмотрим случай когда $x$ лежит на пути 3, другой случай аналогичен. Тогда следующая цепь является искомой:
$$a \mtoa x \mtoc c \to d \mtod b' \to b$$

\mimplies{E}{F}\\
Рассмотрим некоторые три различные вершины $a$, $b$ и $c$. Пусть $(b, b')$ --- любое ребро. Применим E к вершинам $a$, $c$ и ребру $(b, b')$:
$$a \mtoa b \to b' \mtob c$$
Это и есть искомая цепь.

\mimplies{F}{G}\\
Рассмотрим некоторые три различные вершины $a$, $b$ и $c$. Пусть $d$ --- любая другая вершина (если в графе всего три вершины, то это обязательно треугольник и тогда G верно). Применим F к вершинам ($a$, $d$, $b$) и ($b$, $d$, $c$):
$$a \mtoa d \mtob b$$
$$b \mtoc d \mtod c$$
Это простые цепи, значит $b$ не принадлежит ни пути 1, ни пути 4. Тогда следующая цепь является искомой:
$$a \mtoa d \mtod c$$

\mimplies{G}{A}\\
Лучше докажем \mimplies{\overline{A}}{\overline{G}}. 
Пусть есть точка сочленения $w$, удалим её, останется две или больше компонент вершинной двусвязности, пусть $u$ принадлежит первой, $v$ второй. Тогда любая простая цепь из $u$ в $v$ проходит (в исходном графе)  через $w$, то есть существуют три различные вершины, что любая простая цепь соединяющая две из них проходит через третью, а это и есть $\overline{G}$.

\end{document}
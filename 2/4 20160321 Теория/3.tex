\documentclass{article}
\usepackage[utf8]{inputenc}
\usepackage[T2A]{fontenc}
\usepackage[russian]{babel}
\usepackage{amsmath, amssymb}

\title{Задача 3 (семинар 21.03.2016)}
\author{}
\date{}

\newcommand{\olen}{\overline{len}}
\newcommand{\onum}{\overline{num}}
\newcommand{\omaxlen}{\overline{maxlen}}
\newcommand{\osumnum}{\overline{sumnum}}
\newcommand{\logn}{\log n}
\newcommand{\Ologn}{O(\logn)}

\begin{document}

\maketitle
Нам дана последовательность $a[1..n]$.
Обозначим $len_i$ --- длина НВП префикса $a[1..i]$ заканчивающейся в $a_i$, $num_i$ --- число таких НВП. 
$$len_0=0,\ num_0 = 1$$
$$len_i=1+\max\limits_{\substack{j<i\\ a_j<a_i}}(len_j)$$
$$num_i\text{ --- число } j<i, \text{ таких что } a_j<a_i \text{ и } len_j+1=len_i$$
Будем считать $len$ и $num$ используя дерево Фенвика, причём обрабатывать индексы мы будем в порядке возрастания $a_i$. Заведём два массива $maxlen$ и $sumnum$: 
$$maxlen_i=\max\limits_{f(i) \leq j \leq i}(len_j)$$
$$sumnum_i=\sum\limits_{\substack{f(i) \leq j \leq i\\ len_j=maxlen_i}}(num_j)$$
Пересчитывать $maxlen$ и $sumnum$ мы будем следующим образом: пусть у нас обновился $len_i$, тогда для всех $j$, таких что $f(j)\leq i\leq j$ мы увеличиваем $sumnum_j$ на $num_i$ если $maxlen_j=len_i$, в противном случае обновляем $maxlen_j$ и присваиваем $sumnum_j=1$.

Теперь введём ещё два массива, $\olen_i$ --- длина НВП суффикса $a[i..n]$, начинающейся в $a_i$, $\onum_i$ --- число таких НВП. Считаем их аналогично.

Обозначим за $lcs_i$ число НВП длины $l$ проходящих через $a_i$.
$$lcs_i=num_i\cdot \sum\limits_{\substack{i<j\\ a_i<a_j\\ len_i+\olen_j=l}}\onum_j$$

Заметим, что для всех $j>i\ len_i+\olen_j\leq l$
Таким образом, через $a_i$ проходит хотя бы одна НВП $\iff$ $l-len_i=\max\limits_{\substack{i<j\\ a_i<a_j}}(\olen_j)$

Будем считать $lcs$ используя дерево Фенвика. Опять же заведём два массива $\omaxlen$ и $\osumnum$. Обрабатывать индексы мы будем в порядке убывания $a_i$. Итак, обрабатываем индекс $i$. Вначале обновим $\olen_i$ и $\onum_i$, аналогично тому, как мы это делали для $len_i$ и $num_i$. Далее, нас интересует выражение $\sum\limits_{len_i+\olen_j=l}\onum_j$ на суффиксе $a[i+1..n]$. В дереве Фенвика каждый суффикс разбивается на $\Ologn$ непересекающихся отрезков. Пробежимся по этим отрезкам и если $len_i+\omaxlen_j=l$, то прибавим к текущей сумме $\osumnum_j$.

Итак, мы посчитали $lcs_i$, осталось только решить, какие же элементы у нас хорошие. Из предыдущей задачи мы нашли общее число НВП длины $l$, обозначим её $all$. Тогда позиция $i$ --- хорошая $\iff lcs_i=all$.

\end{document}
